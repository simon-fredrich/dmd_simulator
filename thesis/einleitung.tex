\section{Einleitung}

Im Rahmen dieser Bachelorarbeit wird die Interaktion eines Digital micro-mirror devices (DMD) mit kohärentem monochromatischem Licht aus einer Laserquelle. Dazu wird ausgehend von der bereits vorliegenden Veröffentlichung \cite{Lachetta2020.10.02.323527} im Themenbereich eine Simulation mit der Programmiersprache Python geschrieben, welche Daten für den Vergleich mit dem physischen Experiment erzeugt. Auf diese Weise sollen Einblicke in die Thematik erlangt werden, welche die praktische Umsetzung effizienter hinsichtlich Zeit- und Kostenaufwand machen können. Mit der Implementierung können Experimente vor der praktischen Umsetzung getestet werden, wobei z. B. Fehler frühzeitig entdeckt werden können und damit keine Veränderung des experimentellen Aufbaus nötig ist. 
